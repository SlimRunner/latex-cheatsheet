\section{Final Targeted Subjects}
\subsection{Profitability Index (PI)}
The dollar amount you get in return per dollar invested. Not fit to
compare mutually exclusive projects. Suitable for deciding if any number
of projects are good for taking provided you can choose any combination
of them and posses the budget. Basically, you choose them if they are
better than the market value (i.e. PI is more than 1).

\blockquote{It holds that if $\text{NPV}>0\Rightarrow\text{PI}>1$.}

Where PV is present value and $C_0$ is initial investment:

\blockquote{$\text{PI} = \text{PV}~/~C_0$}

If we know the PI of $\text{PI}(Y-X) > 1$ then we can only be certain
that Y is better than X. We have no idea if of any other comparison
including X, Y, \textit{and} the market.

Consider (where CF means cash flow at some time T)

\blockquote{$\text{NPV}_A=\Sigma~\text{CFA}_T/(1+r)^T,~\text{NPV}_B=\Sigma~\text{CFB}_T/(1+r)^T$}

\blockquote{
  $\Rightarrow\text{PI}_A = (\text{NPV}_A+\text{CFA}_0)~/~\text{CFA}_0$ \\
  $\Rightarrow\text{PI}_B = (\text{NPV}_B+\text{CFB}_0)~/~\text{CFB}_0$
}

For independent projects without a budget cap we choose whichever of A
and B that is greater than 1. If they are mutually exclusive we would
instead do

\blockquote{
  let $\text{CFAB}=\text{CFA} - \text{CFB}$ \\
  then $\text{NPV}_{A-B}=\Sigma~\text{CFAB}_T/(1+r)^T$ \\
  so $\text{PI}_{A-B} = (\text{NPV}_{A-B}+\text{CFAB}_0)~/~\text{CFAB}_0$
}

If $\text{PI}_{A-B}>1$ then A is better because for the extra investment
in A relative to B, we are getting more in return than investing in B.
The opposite holds true for $\text{PI}_{B-A}>1$.

\subsection{Capital Budgeting}
Types of cash flows considered:
\begin{itemize}
  \item \textbf{Opportunity Cost} [INCLUDED] forgone cash flows Example: renting unused land.
  \item \textbf{Side Effect} [INCLUDED] taking new project affects existing ones. Example: new product cannibalizing sales.
  \item \textbf{Sunk Cost} [EXCLUDED] past expenses. Example: old research costs.
  \item \textbf{Allocated Cost} [EXCLUDED] expenses from unused assets. Example: unused shared office space.
\end{itemize}
Note that side effects can also be good. Specific cases include
cannibalism, synergy, erosion.

\subsection{Bonds}
A bond consists of a coupon (fixed annual payment), face value (\$1000
in the US), and a return rate or yield to maturity (also fixed). The
price of the bond is affected by the market rate. The fixed quantities
are related by the following equation:

\blockquote{$\text{coupon rate}=\text{coupon} ~/~ \text{face value}$}

Bond price is determined using NVP or annuities since all coupons are
the same

\blockquote{
  $\mathrm{NVP}(r, [0,C_1,C_2,\dots,C_n+\text{Face Value}])$ \\
  Or $\mathrm{An}(r,n,C) + \text{Face Value}/(1+r)^{n}$
}

To find a coupon rate ($x$) you can use the following simplification:

\blockquote{$C\cdot\mathrm{An}(r,n,x) + \text{Face Value}/(1+r)^{n}=P$}

If a regular price is case (1) and there is a default probability $p$ in
case (2), then the present price of the bond is

\blockquote{$(1-p)\mathrm{NPV}_1 + p\mathrm{NPV}_2$}

Once issued a company sells \textit{all} bonds to the market. If it
wishes to re-issue them it must buy the all back.

\subsection{Stock Portfolio}
Where $r_i$ is a single monthly return from a single company

The mean return is: $\overline{r}=\text{avg}(r_i)$

The variance of a return is: $\sigma_r^2 = \frac{1}{t-1}\sum_{i=1}^{t}(r_i-\overline{r})^2$

The standard deviation is: $\sqrt{\sigma_r^2}$

Covariance and correlation:

\blockquote{
  $\mathrm{Cov}(A,B) = \sigma_{A,B} = \frac{\sum(r_{A,t}-\overline{r}_A)(r_{B,t}-\overline{r}_B)}{T-1}$ \\
  $\mathrm{Corr}(A,B) = \rho{A,B} = \frac{\sigma_{A,B}}{\sigma_A\cdot\sigma_B}$
}

For a portfolio (weighted average and weighted variance):
\blockquote{
  $\overline{r}_P=\sum_{i=\text{stocks}}\omega_i\overline{r}_i$ \\
  $\sigma_P^2=\sum_{i}\sum_{j}\omega_i\omega_j\sigma_i\sigma_j\rho_{i,j}=\sum_{i}\sum_{j}\omega_i\omega_j\sigma_{i,j}$
}

Personal reminder of linear algebra simplification where
$M_{\mathrm{Corr}}$ that is a $n\times n$ correlation matrix, and
$M_{\mathrm{coeff}}$ is a $1\times n$ whose only row is $
\{\omega_1\sigma_1, \omega_2\sigma_2, \dots, \omega_n\sigma_n\}$, then

\blockquote{
  $\sigma_P^2 = \mathrm{det}\left(M_{\mathrm{coeff}}\times M_{\mathrm{Corr}}\times M_{\mathrm{coeff}}^{\mathrm{T}}\right)$
}

Also, $\sum_{i=1}^{t}\omega_i = 1$.

\subsection{Payback Discounted NPV}
This is basically computing the NPV of each cash flow and making a
cumulative sum of them to find out when the company breaks even with the
initial investment.

The discounted payback period is always longer than the simple payback
period due to the effect of discounting. The latter is simply the
cumulative sum of the face value of cash without taking into account the
time value of money

\subsection{Break Even Quantity}
Common tool for analyzing the relationship between sales volume and
profitability. There are three common break-even measures
\begin{itemize}
  \item Accounting break-even: sales volume at which net income = 0
  \item Cash break-even: sales volume at which operating cash flow = 0
  \item Financial break-even: sales volume at which net present value = 0
\end{itemize}
$\overline{P}\times Q-\mathrm{FC}-\mathrm{VC}\times Q-\text{Dep}(1-t)-(\mathrm{EAC}-\text{Dep}) = 0$

That is the general formula where EAC is equivalent annual cost, FC is
fixed cost, VC is variable cost, $Q$ is quantity, $\overline{P}$ is
average price, and Dep is depreciation.

A common simplification for \textrm{accounting} break even quantity is:

\blockquote{$Q=(\mathrm{FC}+\text{Dep})~/~(\overline{P} - \mathrm{VC})$}

A common simplification for \textrm{financial} break even quantity is:

\blockquote{$Q = (\mathrm{EAC}-\text{Dep}+(1-t)(\mathrm{FC}+\text{Dep}))~/~((P-\mathrm{VC})(1-t))$}

When $t=0$
\blockquote{$Q = (\mathrm{EAC}+\mathrm{FC})~/~(P-\mathrm{VC})$}
