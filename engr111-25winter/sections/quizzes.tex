\section{Quizzes}
\begin{scriptsize}
  \begin{enumerate}[itemsep=0em]
    \item Which of the following is NOT accurate?
          \begin{itemize}[itemsep=0em]
            \item Capital Structure refers to the relative share of Debt and Equity in a company's Balance Sheet.
            \item \textbf{If a company obtains new debt, then, the Liabilities\&Equity (right) side of the balance sheet would be higher than the Assets (left) side in a Balance Sheet.}
            \item Net Working Capital will deteriorate (go down) if a company buys fixed assets using cash.
          \end{itemize}
    \item Balance Sheet has to be in balance (right and left sides being equal to each other) at the end of each year only.
          \begin{itemize}[itemsep=0em]
            \item False
          \end{itemize}
    \item "The Principal-Agent Problem" refers to the conflict that arises among the C-level executives of a company.
          \begin{itemize}[itemsep=0em]
            \item False
          \end{itemize}
    \item Which of the following is NOT a reason for repurchase of company stock?
          \begin{itemize}[itemsep=0em]
            \item \textbf{To reduce Long Term Debt}
            \item Send a signal to the market due to undervaluation
            \item Not having better investment opportunities
            \item Increase EPS
          \end{itemize}
    \item When a company's stock price goes up in the secondary market, the increase in value is recorded under Equity and under Current Assets.
          \begin{itemize}[itemsep=0em]
            \item False
          \end{itemize}
    \item Most companies would deduct the entire cost of an asset in the income statement for the year of the purchase.
          \begin{itemize}[itemsep=0em]
            \item False
          \end{itemize}
    \item The Current Ratio is the only financial ratio without a set industry standard, with a higher ratio generally indicating better financial health.
          \begin{itemize}[itemsep=0em]
            \item False
          \end{itemize}
    \item Which of the following is NOT accurate?
          \begin{itemize}[itemsep=0em]
            \item Debt to Equity Ratio (D/E) includes all Liabilities whether borrowed or owed as bills that have not been paid yet.
            \item Comparable or higher than industry standard EBIT coupled with lower than industry standard Net Income would invite take-over.
            \item \textbf{All else being equal, the discrepancy between the current ratio and the quick ratio is expected to be greater for Meta than for Target.}
          \end{itemize}
    \item Financial health assessment of a company that has a Debt/Equity ratio that is lower than industry standard would involve:
          \begin{itemize}[itemsep=0em]
            \item A potential holding back on scaling and lost opportunities to grow
            \item Lack of visible discipline in undertaking and paying off debt on time, missing the opportunity to establish responsible track record for future potential debt obligation cost reduction.
            \item Missed opportunity on reducing tax obligation
            \item \textbf{All of the above}
          \end{itemize}
    \item Market capitalization is a more effective metric than balance sheet size for estimating a company's valuation.
          \begin{itemize}[itemsep=0em]
            \item False
          \end{itemize}
    \item From an investor's perspective, Return on Equity (ROE) is a more instructive metric than Profit Margin.
          \begin{itemize}[itemsep=0em]
            \item True
          \end{itemize}
    \item All else being equal, a company with a higher retention ratio can achieve a higher growth rate, particularly if low-cost borrowing opportunities are unavailable.
          \begin{itemize}[itemsep=0em]
            \item True
          \end{itemize}
    \item Everything else being equal, if a positive-profit company does not grow and does not acquire new debt, its D/E ratio is expected to stay the same.
          \begin{itemize}[itemsep=0em]
            \item False
          \end{itemize}
    \item If the External Financing Need (EFN) is negative, the company has surplus funds and does not need to borrow money to achieve its growth target
          \begin{itemize}[itemsep=0em]
            \item True
          \end{itemize}
    \item A company that is growing at the internal growth rate will experience increasing D/E ratio.
          \begin{itemize}[itemsep=0em]
            \item False
          \end{itemize}
    \item If a company is growing according to the "percentage of sales" approach, then, the Total Asset Turnover (Sales/Assets) will go down over time.
          \begin{itemize}[itemsep=0em]
            \item False
          \end{itemize}
    \item If a company is growing at the sustainable growth rate, it is expected that
          \begin{itemize}[itemsep=0em]
            \item The interest payments will not go up
            \item The Capital Structure of the company will change
            \item \textbf{If the profit ad the retention ratios are positive, it is expected that the company acquires new debt.}
          \end{itemize}
  \end{enumerate}
\end{scriptsize}