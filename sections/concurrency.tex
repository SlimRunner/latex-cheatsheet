\section{Concurrency}
\subsection*{Multithreading}
There are three key features of the multithreading model: \textbf{thread
management} which allows each thread to run operations concurrently;
when sharing resources; And, \textbf{message passing} allows different
\textbf{synchronization} which ensures that threads do not interfere
threads to safely send messages to each other.

\subsection*{Async Programming}
In asyncronous programming, statements in a program are not executed
from top to bottom. Instead, a module called a runtime maintains a queue
of coroutines to execute. A coroutine is a function that once run, can
be paused and resumed. The runtime repeatedly dequeues, suspends, and
resumes coroutines resulting in a concurrent execution of each of them.
I/O operations can even be performed outside the queue while a coroutine
is running.

\subsection*{Multithreading vs. Async}
For multithreading
\begin{itemize}
  \item Capable of true parallelism across cores
  \item Preemptive multitasking
  \item Uses more memory; each thread has extensive execution context
  \item Race conditions are common, requiring synchronization
  \item Best suited for CPU bound tasks
  \item Complex error handling
\end{itemize}
For async
\begin{itemize}
  \item Concurrency without true parallelism
  \item Cooperative mulitasking
  \item Uses less memory; each coroutine is lightweight
  \item Race conditions are less common due to single threaded nature
  \item Best suited for IO bound tasks
  \item Easy error handling
\end{itemize}

