\section{Conversion and Casting}
These are similar operations but they are distinct and can be explicit
or implicit.
\begin{itemize}
  \item \textbf{widening} implies going from subtype to supetype.
  \item \textbf{narrowing} may be from supertype to subtype or to a type
  with distinct structure (i.e. unsgined/signed or int/string).
\end{itemize}
Also the relationship between the types matter.

\subsection*{Subtypes and Supertypes}
Formally, $T_{\text{sub}}$ is a subtype of $T_{\text{super}}$ if and
only if $T_{\text{sub}}\subseteq T_{\text{super}}$. That is all values
of the subtype can be represented \textit{exactly} in the supertype.

In C++ int is a subtype of double, but not a subtype of float.

\subsection*{Conversion}
Takes a value of type `A' and creates a \textit{new value} of type `B'.
Often the underlying data structure changes. For example, converting a
float to int.

When a widening conversion is implicit it is called \textit{type
promotion}. When a conversion of any type is implicit it is called a
\textit{coercion}.

\subsection*{Casting}
Takes a value of type `A' and views it as if it was of type `B'. It uses
the same underlying data structure of the source.

A widening cast is called an \textit{upcast}. A narrowing cast is called
a \textit{downcast}.

When the pointer of a subtype is stored in a pointer of a supertype an
upcast happens.
