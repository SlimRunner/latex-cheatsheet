\section{Var Types}
Data can be categorized by its size, range, allowed operations, and more.

Languages use these categories for variable definition, validation,
validation, conversion, casting, polymorphism, generics, etc.

Variables do not inherently have types. They can be \textit{bound} to
types. Values can have types.

\subsection*{User Defined Types}
Types with a custom data structure. They often come in the form of
classes and defining a class usually defines a type. They can have
hierarchies and  form \textbf{supertypes} and \textbf{subtypes}.

Instances of objects are \textbf{value types} and they can also be
referred to indirectly by \textbf{reference types}.

\subsection*{Type Equivalence}
There are two approaches

\textbf{name equivalence} which means that two objects are
type-equivalent if their types have been aliased by the same name.

\textbf{structural equivalence} which means that two objects are
type-equivalent if their structure is identical.