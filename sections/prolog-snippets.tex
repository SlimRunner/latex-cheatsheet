\section{Prolog Snippets}
\subsection*{General Syntax}
\begin{minted}{prolog}
% a fact
functor(at1, at2, ...).
% a rule
functor(at1, at2, ...) :-
    f1(...),f2(...).
\end{minted}
Rules can also be shortcircuited
\begin{minted}{prolog}
func(atom) :- !.
\end{minted}
The negation operator
\begin{minted}{prolog}
func(atom) :- not(...).
\end{minted}

\subsection*{Built-ins}
Functions will be showcased with queries that return true. Their
names should make it clear what they do.
\begin{minted}{prolog}
?- append([1,2],[3,4],[1,2,3,4])
?- sort([4,3,1], [1,3,4])
?- permutation([4,3,1], [3,1,4])
?- reverse([1,2,3],[3,2,1])
?- member(6, [1,6,4])
?- sum_list([4,3,1], 8)
\end{minted}

\subsection*{Lists}
\begin{minted}{prolog}
% normal list
[i1, i2, ...]
% cons
functor([X1, X2, ... | XS])
% equivalent to 
functor(cons(X1, X2, ..., XS))
\end{minted}