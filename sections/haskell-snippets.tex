\section{Haskell Snippets}
\subsection*{Infix/Prefix}
Most operators are infix while function-like operators are prefix.
\begin{minted}{haskell}
-- infix
a + b
a `mod` b
-- prefix
(+) a b
mod a b
\end{minted}

\subsection*{Func, Lambda, \& Currying}
\begin{minted}[fontsize=\scriptsize]{haskell}
fn :: T1 -> T2 -> ... -> Tn -> RTp
fn x1 x2 ... xn = body
\end{minted}
is equivalent to
\begin{minted}[fontsize=\scriptsize]{haskell}
fn :: T1 -> (T2 -> (... -> (Tn -> RTp)))
fn \x1 -> (\x2 -> (... -> (xn -> body)))
\end{minted}
and similar to
\begin{minted}{haskell}
fn = \x1 x2 ... xn -> body
\end{minted}
Note this latter also shows partial function application.

\subsection*{Type Inference}
\begin{minted}[fontsize=\scriptsize]{haskell}
foo :: Int -> Int -> Bool
foo x y = x > (y + 5)
bletch :: [t] -> [[t]]
beltch = map (\x -> [x])
bar ::
  [t] -> [Char] -> (t -> [Char]) -> Bool
bar a b c = head (map c a) > (b ++ "!")
\end{minted}

\subsection*{Conditional}
\begin{minted}{haskell}
if cond1 then "body1"
else if cond2 then "body2"
-- ...
else "otherwise"
\end{minted}

\subsection*{Guards}
\begin{minted}{haskell}
fn params -- no equal here
  | cond1 = "body1"
  | cond2 = "body2"
  -- ...
  | otherwise = "else"
\end{minted}

\subsection*{Lists}
List can be compared with == and comparison is elementwise. Common list
operations are
\begin{minted}{haskell}
null lst -- is list is empty
e1 : lst -- append to front (cons)
e1 : e2 : lst -- and so on
l1 ++ l2 -- list concatenation
head lst -- front of list
tail lst -- back of list
length lst
-- from the front
take n lst -- n-slice of list
drop n lst -- ^ complement
elem e lst -- e in lst
sum lst -- sum reduce
or lst -- or reduce
and lst -- and reduce
-- trims to shortest
zip l1 l2 -- zips lists
cycle lst -- cyclic list
reverse lst
\end{minted}
There are also ranges. Bounds are inclusive unless the step skips them.
\begin{minted}{haskell}
-- assume a <= b
[a..b] -- range from a to b
[b+1..a] -- always []
[a..] -- unbounded list
[a,a+2..b] -- skips a + 1 + 2*n
\end{minted}

\subsection*{List Comprehensions}
\begin{minted}{haskell}
[X | Y1, Y2, ...]
\end{minted}
X is the result of each iteration. Yn is either a list mapping denoted
as \lstinline|Yn <- ln| or a guard denoted as a function that returns
\lstinline|Bool|. Consecutive list mappings are nested not joined.

\subsection*{Pattern Matching}
Define the function multiple times replacing variables with literals,
lists, cons, Type constructors, etc. Haskell will attempt to match a
call to the first matching definition.
\begin{minted}{haskell}
-- literals
f1 1 = "one"
f1 a = a
-- cons (xs may be empty)
f2 [] = "empty"
f2 (x:xs) = "h rest"
f2 (x:y:xs) = "h h rest"
-- Type constructor
data T = 
  A           |
  B Float Int |
  C Int
f3 (A) = "A"
f3 (B a b) = "B"
f3 (C a) = "C"
\end{minted}

\subsection*{Higher Order Funcs}
\begin{minted}[fontsize=\scriptsize]{haskell}
-- mappper
map :: (a -> b) -> [a] -> [b]
-- filter
filter :: (a -> Bool) -> [a] -> [a]
-- reducer
foldl :: (b -> a -> b) -> b -> [a] -> b
foldr :: (a -> b -> b) -> b -> [a] -> b
\end{minted}
Mappers map values from type `a` to type `b`.
\begin{itemize}
  \item first param is a function
  \item second param is a list of values with type `a'
  \item result is a list of values with type `b'
\end{itemize}
Filters exclude elements from list and returns list of same type.
\begin{itemize}
  \item first param is a predicate
  \item second param is the list to be filtered
  \item result is the filtered list
\end{itemize}
Reducers transform a list into something else (often not a list)
\begin{itemize}
  \item first param is the reducer
  \begin{itemize}
    \item first param is the reduced value so far
    \item second param is the one item from the list
    \item result is a value of type `b`
  \end{itemize}
  \item second param is an initial value of type `b`
  \item third param is a list of values of type `a`
  \item result is of type `b`
\end{itemize}

\subsection*{Abstract Data Types}
Composed of a set of Variants
\begin{minted}{haskell}
data T1 = V1 | V2 | ... | Vn
-- can also define members
data T1 =
  V1 Tp1 Tp2 |
  V2 Tq1 Tq2 |
  -- ...
  Vn -- ...
-- or give them names
data T1 =
  V1 { p1 :: Tp1, p2 :: Tp2} |
  V2 { q1 :: Tq1, q2 :: Tq2} |
  -- ...
  Vn -- ...
\end{minted}
Giving names to the members of a Variant allows to patern match them in
different order than defined.

\subsection*{ADT Applications}
Linked list
\begin{minted}{haskell}
data List = 
  Nil | 
  Node {
    val :: Int
    next :: List
  }
\end{minted}
\textit{adding a node to the end of a linked list in Haskell is O(n) because of immutability.}

Binary Search Tree
\begin{minted}{haskell}
data BTree =
  Nil |
  Node {
    val : Int
    left :: BTree
    right :: BTree
  }
\end{minted}
\textit{immutability does not alter complexity of adding node to balanced tree.}

\subsection*{Miscelaneous}
\begin{minted}{haskell}
(-4) -- paren is often required
5 == 6 -- true
5 /= 6 -- false (not equal)
-- [Char] is same as String
-- equiv to f(x, g, y)
f x g y
-- equiv to f(x, g(y))
f x (g y)
\end{minted}
