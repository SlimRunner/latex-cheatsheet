\section*{OOP7}
Given the following hierachies
\begin{minted}{typescript}
interface A { /* impl */}
interface B implements A { /* impl */}
interface C implements A { /* impl */}
class D implements B, C { /* impl */}
class E implements C { /* impl */}
class F implements D { /* impl */}
class G implements B { /* impl */}
\end{minted}
\subsection*{Part A}
We can infer that the super types of each is
\begin{enumerate}
  \item[A]: has no super types
  \item[B]: has one super type: A
  \item[C]: has one super type: A
  \item[D]: has super types: A, B, C
  \item[E]: has super types: A, C
  \item[F]: has super types: A, B, C, D
  \item[G]: has super types: A, B
\end{enumerate}
\subsection*{Part B}
The following function
\begin{minted}{cpp}
void foo(B b) { /* ... */ }
\end{minted}
\noindent
can be called with any of the following argument types:
\begin{itemize}
  \item D
  \item F
  \item G
\end{itemize}

\subsection*{Part C}
The following code
\begin{minted}{cpp}
void bar(C c) { /* ... */ }

void bletch(A a) {
  bar(a);
}
\end{minted}
\noindent
would not compile because \codebox{A} is a super type of \codebox{C},
but \codebox{C} is not a super type of \codebox{A}

\section*{OOP8}
\textbf{Inheritance} is the concept of sharing behavior and structure in
such a way that a class of some sort serves as a blueprint to create
instances of objects that manipulate data in an ordered manner.

\textbf{Subtype Polymorphism} is a way to taking advantage of
inheritance to access different objects through their common traits.
More specifically to use an specific type as if it were one of its
subtypes.

\textbf{Dynamic Dispatch} is a method of resolving function calls of
similar names when they can be accessed through different subtypes of
the same object. This has to be done at runtime because it cannot be
determined at compile time which subtype the object will be represented
as before the program runs. Therefore, something called v-table is
attached to each class and since every pointer holds a copy of the class
blueprint it allows the pointers to tell the compiler at runtime what
pointer to use to perform a method call. Only functions that need to be
overriden are included in the v-table.

\section*{OOP9}
Subtype polymorphism cannot be used in dynamically typed languages (DTL)
because it relies on determining the type of objects to act as if they
were of another type. However, in DTLs variables do not have type so no
such inferences can be made from objects alone. Something similar takes
place which is duck typing, but that is not subtype polymorphism.

Dynamic dispatch on the other hand can happen in DTLs. However, since
variables do not have type we cannot rely on the class to give
information to the program about where to find the functions. Instead
each \textit{object} stores this information when it is created.
\begin{minted}{python}
class Animal:
  def make_noise():
    raise NotImplementedError()

class Duck(Animal):
  def make_noise():
    print("quack")

class Dog(Animal):
  def make_noise():
    print("woof")

def make_it_talk(animal):
  if isinstance(animal, Animal)
    animal.make_noise()
  else:
    raise ValueError()

duck = Duck()
dog = Dog()

make_it_talk(duck) # prints "quack"
make_it_talk(dog) # prints "woof"
\end{minted}
\noindent
The above example is not using duck typing because this is not about
finding a function that matches the call, but disambiguating which
function we are meaning to call. Both duck and dog are animals but
neither raised \codebox{NotImplementedError} because each object held in
its v-table the correct pointer to their respective
\codebox{make_noise}.

\section*{PRLG5}
I will place in a comment whether it unifies or not and in the list
following the reasoning to avoid clutter.
\begin{minted}{prolog}
foo(bar,bletch) with foo(X,bletch)                 % (1)
foo(bar,bletch) with foo(bar,bletch,barf)          % (2)
foo(Z,bletch) with foo(X,bletch)                   % (3)
foo(X, bletch) with foo(barf, Y)                   % (4)
foo(Z,bletch) with foo(X,barf)                     % (5)
foo(bar,bletch(barf,bar)) with foo(X,bletch(Y,X))  % (6)
foo(barf, Y) with foo(barf, bar(a,Z))              % (7)
foo(Z,[Z|Tail]) with foo(barf,[bletch, barf])      % (8)
foo(Q) with foo([A,B|C])                           % (9)
foo(X,X,X) with foo(a,a,[a])                       % (10)
\end{minted}
\begin{enumerate}
  \item \textbf{Unifies} because \textbf{X} $\to$ \textbf{bar} and bletch matches
  \item \textbf{Does not unify} because arity does not match
  \item \textbf{Unifies} because \textbf{Z} $\lrarr$ \textbf{X} and bletch matches
  \item \textbf{Unifies} because \textbf{X} $\to$ \textbf{barf} and \textbf{Y} $\to$ \textbf{bletch}
  \item \textbf{Does not unify} because \textbf{bletch} does not match \textbf{barf}
  \item \textbf{Unifies} because \textbf{foo} arity matches, \textbf{X} $\to$ \textbf{bar} (in both instances), \textbf{bletch} arity matches, and \textbf{Y} $\to$ \textbf{barf}
  \item \textbf{Unifies} because barf matches and \textbf{Y} $\to$ \textbf{bar(a,Z)}
  \item \textbf{Does not unify} because \textbf{Z} $\to$ \textbf{barf} and \textbf{Z} $\to$ \textbf{bletch} is a contradiction
  \item \textbf{Unifies} because \textbf{Q} $\to$ \textbf{[A,B|C]}
  \item \textbf{Does not unify} because \textbf{X} $\to$ \textbf{a} and \textbf{X} $\to$ \textbf{[a]} is a contradiction
\end{enumerate}

\section*{CNTL2}
The provided code:
\begin{minted}{python}
class Node:
  def __init__(self, val):
    self.value = val
    self.next = None

class HashTable:
  def __init__(self, buckets):
    self.array = [None] * buckets

  def insert(self, val):
    bucket = hash(val) % len(self.array)
    tmp_head = Node(val)
    tmp_head.next = self.array[bucket]
    self.array[bucket] = tmp_head
\end{minted}

\subsection*{Part A}
For the generator
\begin{minted}{python}
class Node:
  pass # same

class HashTable:
  def __init__(self, buckets):
    pass # same

  def hashGen(self):
    for node in self.array:
      while node:
        yield node.value
        node = node.next

  def __iter__(self):
    return self.hashGen()

  def insert(self, val):
    pass # same
\end{minted}
\noindent
It a little cheaty because I technically used an iterator---namely
returned an array from \codebox{__iter__}---but the question did not
prohibit that.

\subsection*{Part B}
For the iterator class
\begin{minted}{python}
class HashIter:
  def __init__(self, root):
    self.__iter = iter(root)
    self.__node = next(self.__iter)

  def __next__(self):
    while self.__node is None:
      self.__node = next(self.__iter)
    while self.__node is not None:
      node = self.__node
      self.__node = node.next
      return node.value
    raise StopIteration()

class Node:
  pass # same

class HashTable:
  def __init__(self, buckets):
    pass # same

  def __iter__(self):
    return HashIter(self.array)

  def insert(self, val):
    pass # same
\end{minted}
\noindent
Then again I used an iterator, but the prompt did not say I cannot
iterate over the list using Python facilities.

\subsection*{Part C}
Use the iterator
\begin{minted}{python}
# assume htable = HashTable(...)
# assume htable has been populated
for node in htable:
  print(node) # print node values
\end{minted}

\subsection*{Part D}
Show how it works
\begin{minted}{python}
# assume htable = HashTable(...)
# assume htable has been populated
h_it = htable.__iter__()
while True:
  try:
    i = h_it.__next__()
  except StopIteration:
    break
  print(i)
\end{minted}

\subsection*{Part E}
It can essentially mimic the code from the generator except it mutates
the node value instead of yielding it.
\begin{minted}{python}
# inside HashTable
def forEach(self, functor):
  for node in self.array:
    while node:
      functor(node.value)
      node = node.next
\end{minted}