\section{Garbage Collection}
Garbage collection (GC) is the automatic reclamation of memory which was
allocated by a program, but is no longer referenced in code.
\begin{itemize}
  \item It eliminates memory leaks. This ensures memory allocated for
  objects is freed once it’s no longer needed
  \item It eliminates dangling pointers and the use of dead objects.
  This prevents access to objects after they have been de-allocated
  \item It eliminates double-free bugs. This eliminates inadvertent
  attempts to free memory more than once
  \item It eliminates manual memory management. This simplifies code by
  eliminating manual deletion of memory
\end{itemize}
Note that GCs are non-deterministic.

\subsection*{Mark and Sweep}
It perdiodically checks what memory is still accessible (when there is
enough \textbf{memory pressure}). The criteria are:
\begin{itemize}
  \item \textbf{root objects} is a global variable, local variables and
  parameters (in all stack frames)
  \item \textbf{reachable from a root object} likely in the heap refered
  by a root object
\end{itemize}

It uses a queue to store the references. A bit is used to mark variables
which are in use.

After the marking is done, unmarked objects are swept. In order to
organize the memory in the heap a linked list is used to keep track of
all references.

At the sweep step adjacent blocks of free memory are coalesced. This
method is susceptible to memory fragmentation.

\subsection*{Mark and Compact}
The mark step is similar to Mark and Sweep. However instead of sweeping,
a new chunk of memory large enough to allocate all the marked objects in
a compact chunk. This prevents fragmentation, but reduces the amount of
effective memory by half. All pointers must be updated and then the old
memory chunk is cleared.

\subsection*{Referece Counting}
Every object has a count of how many references point to it. At every
assignment and end-of-scope the refrences are updated.

Upon destruction of an object falling to a reference count of zero, all
its references must be reduced accordingly.

Destruction may happen immediately, but most often 0-ref objects are
stored in a list of objects to delete later. Reference counting is also
suceptible to memory fragmentation.

This strategy fails when there are \textbf{cyclic references}.

\subsection*{Ownership Model}
It is based on the fact that an object may own another (one or many). In
some implementations there can only be one owner. When the ownership
changes the previous one is invalidated. Rust ownership model is based
on this. It essentially alters the lifetime of variables.

Rust builds on this model further by adding a \textit{borrow checker}.
It allows other variables to borrow the ownershipt of an oject. This
makes it impossible to mutate the original object, but it can access the
objects data.

C++ ownership model is implemented as an STL container. They are called
smart pointers. These are objects that point to a reference count and
the object itself. They do \textit{not} know of each other existance
other than the reference count.

The reference counter of a shared pointer in C++ is a reference, not a
primitive value. This allows multiple instances to update the counter in
different scopes.

\subsection*{Object End-Of-Life}
In C++ we have something called a \textbf{destructor} which is in charge
of destroying the object. It is \textit{always} called when the object
lifetime ends.

Most other languages with a GC have something called \textbf{Finalizer
Methods}. Some other languages have methods to manually dispose of the
objects, but does not expose the cleanup dynamics.

Unlike a destructor, a finalizer is \textit{not} in charge of freeing
memory. It ends other resources such as file handles or network
connections. Also, just like GCs, finalizers may never be called.

Example of manual methods are C\# `IDisposable' interface which allows to
write a function to free non-memory resources called `Dispose()'. This
function is not called automatically by the GC unlike a finalizer.