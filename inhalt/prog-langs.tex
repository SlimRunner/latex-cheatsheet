\section{Programming Languages}
\subsection*{Building Blocks \& Dimensions}
These are some traits that makes a prog. language distinct:
\begin{itemize}
  \item type systems
  \item parameter passing
  \item scoping
  \item memory management
\end{itemize}
Some other considerations are
\begin{itemize}
  \item variable declaration/assignment
  \item rules for implicit conversions
  \item container bound checks
\end{itemize}
Finally, all of these can be categorized into:
\begin{itemize}
  \item \textbf{syntax}: legal ways to write code in the language. This
  is to code what grammar is to human language.
  \item \textbf{semantics}: \textit{meaningful} was to write code in the
  langauge. A sentence in English may be correct, but does it make
  sense?
\end{itemize}

There are compiled and interpreted languages.
% These are broad categories that in the real world overlap, but in
% general if a \textit{binary} is generated which does not change
% directly as a result of changing the source code then it is said to be
% compiled.

\subsection*{Compilation}
Starts with \textit{source code} which is used by \textbf{Lexer} to
generate \textit{Lexical Units} which are used by \textbf{Parser} to
generate an \textit{Abstract Syntax Tree} which is used by
\textbf{Semantic Analyzer} to generate an \textit{Annotated Parse Tree}
which is used by \textbf{Intermediate Representation Generator (IRG)} to
generate an \textit{Intermediate Representation (IR)} which is used by
\textbf{Code Generator} to generate \textit{Machine Code or Byte Code}
which is put together by a \textbf{linker} into a \textit{binary} or
\textbf{interpreted} instruction by instruction, normally, from the byte
code.

\textit{Quick Facts}: a pretty commonly used IR is LLVM; the process of
compiling a compiler for the language you are creating is called
bootstrapping. In the real world compilation and interpreting overlap
especially, but not exclusively with just-in-time (JIT) compilers.