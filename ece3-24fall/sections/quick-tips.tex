\section{Quick Tips}
\subsection*{Kirchhoff Current Law (KCL)}
Always assume currents are going out for \textit{all} nodes in the same
circuit. Subtract voltage at node from voltage after the first component
that changes it. If the whole branch (i.e. section with the same
current) does not change voltage then make up a direction, label, and
add it to the equation.

\subsection*{Kirchhoff Voltage Law (KVL)}
Try to avoid. If not possible do KCL first. Label all the branches and
make up a current direction. Do all the loops clockwise (for ease). The
chosen current directions determine where are voltage changes (+ to - or
- to +). If the loop encounters + to - then make the term negative and
make it positive otherwise.

\subsection*{Transients Cap/Ind}
After a long time.
\begin{itemize}
  \item \textbf{capacitors} $\equiv$ \textbf{open circuit}
  \item \textbf{inductors} $\equiv$ \textbf{short circuit}
\end{itemize}
After a sudden switch flip you can assume all the inductors have the
exact same current thru, and that all capacitors still have the exact
same voltage across.

\subsection*{Op Amps}
Most of the time you want to set up KCL on the negative ($v_n$) and
positive ($v_p$) nodes of the op amp input. Remember that ideal om amps
NOT draw any current (can be ignored for KCL) and the voltage goes
through from positive to negative unimpeded (all it does is measure the
voltage to use in a VCVS). It will never output more voltage than the
rails (usually not present in problems).
