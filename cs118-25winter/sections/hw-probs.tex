\section{Homework Examples}
\subsection{Utilization}
Using the topology in section \ref{ssec:delaytable} node A sends a file
of infinite size to node B. The packet size is \qty{1000}{\byte}. The
ACK is of negligible size so has 0 tranmission delay.

Using stop-and-wait for a single packet from $A\to R_1$ trans is \qty{1}{\milli\second} prop
is \qty{2}{\milli\second}, from $R_1\to R_2$ trans is
\qty{4}{\milli\second} prop is \qty{4}{\milli\second}, from $R_2\to B$
trans is \qty{1}{\milli\second} prop is \qty{2}{\milli\second}.

Total trans time is $1+4+1=6$ and total prop time is $2+4+2=8$. Total
time for ACK is just the prop time since it has negligible size. Total
time to wait for the ACK is $6+8\cdot2 = \qty{22}{\milli\second}$.

Utilization is at $A\to R_1$ transmission which is
\qty{1}{\milli\second}. Therefore, utilization is $\frac{1}{22}\approx
4.5\%$.

Now assume that we use Go-Back-N. There is no retrans timer the R nodes
have finite, but really large packet buffers. Links have no loses. What
is the max possible throughput without loses (remember that overflowing
the buffer will cause packet drop). We will use the bottleneck to
compute the sliding window. Total in flight time is
\qty{22}{\milli\second}. The bottleneck is at $R_1\to R_2$ and all that
matters is the transmisison delay which is \qty{4}{\milli\second}.
Therefore, sliding window can be of size $\lceil\frac{22}{4}\rceil=6$
which is also the max \# of in-flight packet.