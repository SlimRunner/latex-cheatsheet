\section{Polymorphism}
Allows to operate on a subtype as if it was one of its supertypes.

Class inheritance and generic programming are two examples of
polymorphism. In static typed language there are
\begin{itemize}
  \item \textbf{Ad-hoc polymorphism}: essentially using function
  overloading to operate differently on different types.
  \item \textbf{Subtype polymorphism}: a function that can work on a
  base class and by extension any derived class.
  \item \textbf{Parametric polymorphism}: essentially a generic function
  that can handle multiple, possibly unrelated, types.
\end{itemize}

In un-typed languages, such as Python, \textbf{ad-hoc} is \textit{not}
possible because overloading does not work in languages without function
signatures. However, parametric polymorphism is possible through duck
typing.

\subsection*{Parametric Polymorphism}
Can be implemented as \textbf{templates} or \textbf{generics}.

\textbf{Templates} do almost all of the work at compile-time, and
generate a concrete version of each function needed. Once templates
finish resolving the result is code that could have been written by
hand.

\textbf{Generics} Instead of generating a new function for each
parametrized type, it compiles just one ``version'' of the generic
function or class---independent of the types that actually use the
generics function.

It requires that the function is \textit{universal}. Meaning that any
type can be used in place of the generic type. In order to give
flexibility languages allow to \textit{type bound} the generics to an
arbitrary subset of subtypes.