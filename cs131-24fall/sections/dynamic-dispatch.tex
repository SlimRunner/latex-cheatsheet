\section{Dynamic Dispatch}
It is the process of figuring out which method to call when a call is
made to a virtual method at runtime. For example, when the subtype
overrides a method from its supertype; calling that method through a
reference type makes it non-trivial to determine which method to call at
compile time.

In \textbf{statically typed} languages dynamic dispatch is implemented
as a pointer to a table that the program can consult at runtime when
accessing an instance of that class type. This is called a virtual table
(aka vtable).

In \textbf{dynamically typed} langauges dynamic dispatch is implemented
as a \textit{member} of each instance that stores the vtable.

For reference the opposite of dynamic dispatch is called \textit{static
dispatch}.