\section*{PRLG3}
Considering a bunch of predicates for \codebox{color(_,_)},
\codebox{food(_)}, and \codebox{likes(_,_)}.

\subsection*{Part A}
In prose we can say that person P likes item F, F is food, and F is
color red. In prolog using the predicates of the facts we can write:
\begin{minted}{prolog}
likes_red(P) :- likes(P,F),food(F),color(F,red).
\end{minted}

\subsection*{Part B}
In prose we can say that a person P likes item F1 which is food of color
C, and menachen likes item F2 which is also food, and is of color C. In
prolog using the predicates of the facts we can write:
\begin{minted}{prolog}
likes_foods_of_colors_that_menachen_likes(P) :-
    likes(P, F1),
    likes(menachen, F2),
    food(F1),
    food(F2),
    color(F1, C),
    color(F2, C).
\end{minted}

\section*{PRLG4}
Considering the following
\begin{minted}{prolog}
road_between(la, seattle).
road_between(la, austin).
road_between(seattle, portland).
road_between(nyc, la).
road_between(nyc,boston).
road_between(boston,la).
\end{minted}
\noindent
We can write a prolog rule which determines if two cities are reachable
to each other. There is something weird with the example. It forms a
strongly connected component so for testing purpose it is not good to
test if two cities cannot reach each other. Unless it wasn't meant to be
general. Anyway, I could not get rid of the infinite loop, but this
technically determines if two cities can reach each other. It just hangs
forever if no such path exists.
\begin{minted}{prolog}
reachable(C1, C2) :- road_between(C1, C2).
reachable(C1, C2) :- road_between(C2, C1).
reachable(C1, C2) :- reachable(C1, C3), reachable(C3, C2).
reachable(C1, C2) :- reachable(C2, C3), reachable(C3, C1).
\end{minted}

\section*{PRLG6}
Filling in the blanks
\begin{minted}{prolog}
% adds a new value X to an empty list
insert_lex(X, [], [X]).

% the new value is < all values in list
insert_lex(X, [Y | T], [X, Y | T]) :- X =< Y.

% adds somewhere in the middle
insert_lex(X, [Y | T], [Y | NT]) :-
  X > Y, insert_lex(X, T, NT).
\end{minted}
\noindent
We can think about this one in reverse. What if three atoms (no
variables)? Then, we can quickly realize that the third function will
truncate both lists until the first number is greater or equal to the
head of both lists. At such point it will call either the first or
second rule. In case of the first it that the first argument was greater
than all numbers. Otherwise, the second list checks if the first
argument is in order (or in fact a member of the third argument), and
return true or false.

\section*{PRLG7}
\begin{minted}{prolog}
% count_elem(List, Accumulator, Total)
% Accumulator must always start at zero
count_elem([], Total, Total).
count_elem([Hd | Tail], Sum, Total) :-
  Sum1 is Sum + 1,
  count_elem(Tail, Sum1, Total).
\end{minted}
\noindent
Similarly, we can verify this one by reasoning in reverse. That is by
passing a list, a zero and a number (instead of a variable). Now we can
clearly see that the recursive function is simply adding 1 to the
accumulator until it equals the total.

\section*{PRLG8 (optional)}
\begin{minted}{prolog}
% number should always be positive
gen_list(_, 0, []) :- !.
gen_list(X, N, [X | T]) :-
  Nsub is N - 1, gen_list(X, Nsub, T).
\end{minted}
\noindent
This is similar to problem PRLG7 except that we are subtracting, and the
pattern matching ensures that all the items in the list are equal.

\section*{PRLG9}
\begin{minted}{prolog}
append_item([], Item, [Item]).
append_item([X | T], Item, [X | NT]) :-
  append_item(T, Item, NT).
\end{minted}
\noindent
Thinking in reverse makes things so much easier. Anyway, also this was
basically problem PRLG6 without the ``insert in the middle'' step.
