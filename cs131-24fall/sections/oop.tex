\section{OOP}
A class defines:
\begin{itemize}
  \item a public interface
  \item code for methods
  \item data fields
\end{itemize}

An object (aka instance) has
\begin{itemize}
  \item an interface
  \item code
  \item field values
\end{itemize}

\subsection*{Methods}
Methods are functions that operate on a class, usually by having the
instance passed as a hidden or explicit parameter. Common identifiers
for this instance parameter are `this' and `self'.

\subsection*{Property}
It can be an accessor (read only), or a mutator (changes state). These
are purpose specific methods that facilitate interaction with a class
(does not need parenthesis).

A rule of thumb is that if a method uses minor computation it can be
a property. Otherwise, a method is more appropriate.