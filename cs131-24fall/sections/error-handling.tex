\section{Error Handling}
Errors can come from many sources: bugs, unrecoverable errors,
recoverable errors, and others.

Some ways to handle them are

\subsection*{Error Objects}
Error objects never contain a valid value. They often also contain a
message about the error. These objects can also be nested to represent a
chain of failure events.

One alternative strategy to implement this is to return a tuple from a
function call which contains the actual result and the error object (or
null).

\subsection*{Result Objects}
These wrap the result of a function within an object that stores
information whether the value is valid or not. In some languages such as
Rust trying to access an invalid values leads to a panic.

\subsection*{Optional Objects}
Similar to result objects, but they do not provide details for the
failure.

\subsection*{Assertions}
These let execution continue if they can verify their assertion is true.
Typically, the following are used to verify
\begin{itemize}
  \item \textbf{preconditions}: something that must be true before the
  code continues
  \item \textbf{postconditins}: something that must be true after the
  code did something
  \item \textbf{invariant}: for example, loop invariants where something
  must hold for each iteration
\end{itemize}

\subsection*{Exception Handling}
This is used to handle errors that are recoverable. Often it is directly
supported by the language such as a Try-Catch clause. When an exception
happens no more instructions will be executed, and the call stack
becomes undone frame by frame until a try catch with a matching catch
for the exception raised is found. Otherwise, the error keeps
propagating until the program exits.

These must be \textit{exceptional} situations, so they should be rare.

Some languages also include a third clause called ``finally block''
which is always run regardless of what the intended execution path is.
It will even supercede any return statement anywhere else in the
try-catch block. Exception handling can also provide some guarantees.
\begin{itemize}
  \item \textbf{No-Throw Guarantee}: a function guarantees that it will
  not throw an error
  \item \textbf{Strong Exception Guarantee}: ensures the state of the
  program is reverted to a valid one
  \item \textbf{Basic Exception Guarantee}: leaves the program in a
  valid state (not necessarily as before but valid)
\end{itemize}

\subsection*{Panics}
These are unrecoverable errors.When they happen the program will crash.
They contain \textbf{an error message}, \textbf{a stack trace}, and
context of why this happened.