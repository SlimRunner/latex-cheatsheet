\section*{HASK7}
\subsection*{Part A}
The most important part in this function was the guard for the list of
numbers. We essentially filter out the numbers which do not divide n
evenly. Also the instructions did t. The final code is:
\begin{minted}{haskell}
all_factors :: Integer -> [Integer]
all_factors n = [x | x <- [1..n], mod n x == 0]
\end{minted}

\section*{HASK8}
This is a textbook Bellman equation type of problem. Essentially, there
are two recursive calls. If the heads do not match we just keep checking
the next item in a1. If, on the other hand, they do match we must add
the case in which the head of a1 is included and when it is excluded.
The final code is:
\begin{minted}[fontsize=\scriptsize]{haskell}
count_occurrances :: [Int] -> [Int] -> Int
count_occurrances [] a2 = 1
count_occurrances a1 [] = 0
count_occurrances (x1 : xs) (y1 : ys)
  | x1 == y1 = count_occurrances xs ys + count_occurrances (x1 : xs) ys
  | otherwise = count_occurrances (x1 : xs) ys  
\end{minted}
One important thing to highlight here is that the order of the 1
basecase must come before the 0 base case; otherwise, we would get the
wrong answer for two empty lists.

\section*{HASK10}
\subsection*{Part B}
Use the filter and all functions to write a Haskell function named
\codebox{only_odds} that takes in a list of Integer lists, and returns
all lists in the input list that only contain odd numbers (in the same
order as they appear in the input list)
\begin{minted}{haskell}
only_odds ss = filter (\s -> (all (\i -> mod i 2 == 1) s)) ss
\end{minted}
the following is equivalent
\begin{minted}{haskell}
only_odds :: [[Integer]] -> [[Integer]]
only_odds = filter (all (\i -> mod i 2 == 1))
\end{minted}

\subsection*{Part C}
Use one of foldl or foldr and the \codebox{largest} function
to write a Haskell function named \linebreak\codebox{largest_in_list} that takes
in a list of Strings and returns the longest String in the list. If the
list is empty, return the empty string. If there are multiple strings
with the same maximum length, return the one that appears first in the
list. Do not use the map, filter or maximum functions in your answer.
\begin{minted}{haskell}
largest :: String -> String -> String
largest first second =
  if length first >= length second then first else second

largest_in_list :: [String] -> String
largest_in_list (x1 : xs) = foldl largest x1 xs
\end{minted}

\section*{HASK11}
\subsection*{Part A}
We have three cases to take care of. The empty list, the list with one
element, and the list with many entries. We can use a simple
if-expression to assign sum the values.
\begin{minted}{haskell}
count_if :: (a -> Bool) -> [a] -> Int
count_if pred [] = 0
count_if pred [x] = if pred x then 1 else 0
count_if pred (x : xs) = 
  (if pred x then 1 else 0) + count_if pred xs
\end{minted}

\subsection*{Part B}
This is pretty easy we can simply take the length of the filtered list.
\begin{minted}{haskell}
count_if_with_filter :: (a -> Bool) -> [a] -> Int
count_if_with_filter pred list = length (filter pred list)
\end{minted}

\subsection*{Part C}
We need to simply add the accumulator to the result of counting or not
each element in the list.
\begin{minted}{haskell}
count_if_with_fold :: (a -> Bool) -> [a] -> Int
count_if_with_fold pred =
  foldl (\y x -> y + (if pred x then 1 else 0)) 0
\end{minted}
I also took a suggestion from the linter to omit the list parameter and
let the PFA take care of the rest.

\section*{HASK14}
\subsection*{Part C}
Consider the following function
\begin{minted}{haskell}
foo :: Integer -> Integer -> Integer -> (Integer -> a) -> [a]
foo x y z t = map t [x,x+z..y]
\end{minted}
When curried it becomes
\begin{minted}[fontsize=\scriptsize]{haskell}
foo :: Integer -> (Integer -> (Integer -> (Integer -> a) -> [a]))
foo = \x -> (\y -> (\z -> (\t -> map t [x,x+z..y])))
\end{minted}

\section*{HASK15}
\subsection*{Part A}
Define a new Haskell type \codebox{InstagramUser} that has two value
constructors (without parameters), Influencer and Normie.
\begin{minted}{haskell}
data InstagramUser = Influencer | Normie
\end{minted}

\subsection*{Part B}
Write a function named \codebox{lit_collab} that takes in two
InstagramUsers and returns True if they are both Influencers and False
otherwise.
\begin{minted}{haskell}
lit_collab :: Instagram_user -> Instagram_user -> Bool
lit_collab Influencer Influencer = True
lit_collab a b = False
\end{minted}

\subsection*{Part C}
Modify your \codebox{InstagramUser} type so that the Influencer value
constructor takes in a list of Strings representing their sponsorships.
\begin{minted}{haskell}
data InstagramUser = Influencer [String] | Normie
\end{minted}

\subsection*{Part D}
Write a function \codebox{is_sponsor} that takes in an \codebox{InstagramUser} and
a String representing a sponsor, then returns True if the user is
sponsored by sponsor (this function always returns False for Normies).
\begin{minted}[fontsize=\scriptsize]{haskell}
is_sponsor :: InstagramUser -> String -> Bool
is_sponsor Normie _ = False
is_sponsor (Influencer sponsors) entity =
  foldl (\ acc sponsor -> acc || sponsor == entity) False sponsors
\end{minted}

\subsection*{Part E}
Modify your \codebox{InstagramUser} type so that the Influencer value
constructor also takes in a list of other \codebox{InstagramUser}s
representing their followers (after their sponsors).
\begin{minted}[fontsize=\scriptsize]{haskell}
data InstagramUser = Influencer [String] InstagramUser | Normie
\end{minted}

\subsection*{Part F}
Write a function \codebox{count_influencers} that takes in an
\codebox{InstagramUser} and returns an Integer representing the number
of Influencers that are following that user (this function always
returns 0 for Normies).
\begin{minted}{haskell}
count_influencers :: InstagramUser -> Integer
count_influencers Normie = 0
count_influencers (Influencer sponsors followers) =
  sum (map is_influencer followers)
  where
    is_influencer :: InstagramUser -> Integer
    is_influencer Normie = 0
    is_influencer _ = 1
\end{minted}

\section*{HASK17}
\subsection*{Part A}
Using C++, write a function named longestRun that takes in a vector of
booleans and returns the length of the longest consecutive sequence of
true values in that vector.
\begin{minted}{cpp}
size_t longestRun(const std::vector<bool> & bseq) {
  size_t total = 0, max = 0;
  for (auto const & term: bseq) {
    total = (total + term) * term;
    max = max < total ? total : max;
  }
  return max;
}
\end{minted}

\subsection*{Part B}
Now write the equivalent function in Haskell
\begin{minted}[fontsize=\scriptsize]{haskell}
longestRun :: [Bool] -> Int
longestRun l = foldl max 0 (foldr stride [0] l)
  where
    stride v (r : rs) = if v then r + 1 : rs else 0 : (r : rs)
\end{minted}

\subsection*{Part C}
Using C++, write a function named maxTreeValue that takes in a Tree
pointer root and returns the largest value within the tree. If root is
nullptr, return 0. This function may not contain any recursive calls
(solution: use a stack or queue).
\begin{minted}{cpp}
unsigned maxTreeValue(Tree *root) {
  if (!root) return 0;
  std::vector<Tree *> stack{root};
  unsigned maxval = root->value;
  for (Tree *node = nullptr; stack.size();) {
    node = stack.back(); stack.pop_back();
    maxval = maxval < node->value ? node->value : maxval;
    for (const auto & children: node->children) {
      stack.push_back(children);
    }
  }
  return maxval;
}
\end{minted}

\subsection*{Part D}
Write the same function using Haskell.
\begin{minted}{haskell}
maxTreeValue :: Tree -> Integer
maxTreeValue Empty = 0
maxTreeValue (Node val next) =
  foldl (\a v -> max a (maxTreeValue v)) val next
\end{minted}

\section*{HASK19}
\subsection*{Part A}
This section showcases tail recursion optimization in part B.
\begin{minted}{haskell}
sumSquares :: Integer -> Integer
sumSquares 0 = 0
sumSquares n = n * n + sumSquares (n - 1)
\end{minted}

\subsection*{Part B}
\begin{minted}{haskell}
sumSquares :: Integer -> Integer
sumSquares n = n = helper n 0
where
  helper 0 acc = acc
  helper n acc = helper (n - 1) (acc + n * n)
\end{minted}