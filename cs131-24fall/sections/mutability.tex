\section{Mutability}
If something is immutable it means its internal data cannot change.
Essentially it is read only.

Using immutability can improve code quality and help prevent
\begin{itemize}
  \item aliasing bugs
  \item identity bugs
  \item multithreading bugs
  \item temporal coupling bugs
\end{itemize}

\subsection*{Class Immutability}
This is a feature of a language that prevents data in classes from
changing once it has been instantiated. If a language has class
immutability it \textit{cannot} be optional.

\subsection*{Object Immutability}
Similar to class immutability except that the programmer can decide
which specific instances of the class are mutable.

\subsection*{Assignment Immutability}
It prevents re-assignment to the bound variable. Methods may still
mutate the instance.

\subsection*{Reference Immutability}
It prevents mutations through a reference. For example, in C++
\lstinline|const &|.
